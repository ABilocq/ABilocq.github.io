%%%%%%%%%%%%%%%%%
% This is an sample CV template created using altacv.cls
% (v1.7, 9 August 2023) written by LianTze Lim (liantze@gmail.com). Compiles with pdfLaTeX, XeLaTeX and LuaLaTeX.
%
%% It may be distributed and/or modified under the
%% conditions of the LaTeX Project Public License, either version 1.3
%% of this license or (at your option) any later version.
%% The latest version of this license is in
%%    http://www.latex-project.org/lppl.txt
%% and version 1.3 or later is part of all distributions of LaTeX
%% version 2003/12/01 or later.
%%%%%%%%%%%%%%%%

%% Use the "normalphoto" option if you want a normal photo instead of cropped to a circle
% \documentclass[10pt,a4paper,normalphoto]{altacv}

\documentclass[10pt,a4paper,ragged2e,withhyper]{altacv}
%% AltaCV uses the fontawesome5 and packages.
%% See http://texdoc.net/pkg/fontawesome5 for full list of symbols.

% Change the page layout if you need to
\geometry{left=1.25cm,right=1.25cm,top=1.5cm,bottom=1.5cm,columnsep=1.2cm}

% The paracol package lets you typeset columns of text in parallel
\usepackage{paracol}
\usepackage{hyperref}

% Change the font if you want to, depending on whether
% you're using pdflatex or xelatex/lualatex
% WHEN COMPILING WITH XELATEX PLEASE USE
% xelatex -shell-escape -output-driver="xdvipdfmx -z 0" sample.tex
\ifxetexorluatex
  % If using xelatex or lualatex:
  \setmainfont{Roboto Slab}
  \setsansfont{Lato}
  \renewcommand{\familydefault}{\sfdefault}
\else
  % If using pdflatex:
  \usepackage[rm]{roboto}
  \usepackage[defaultsans]{lato}
  % \usepackage{sourcesanspro}
  \renewcommand{\familydefault}{\sfdefault}
\fi

% Change the colours if you want to
\definecolor{SlateGrey}{HTML}{2E2E2E}
\definecolor{LightGrey}{HTML}{666666}
\definecolor{DarkPastelRed}{HTML}{124076}
\definecolor{myRed}{HTML}{8F0D0D}
\definecolor{myBlue}{HTML}{00215E}
\definecolor{PastelRed}{HTML}{00215E}
\definecolor{GoldenEarth}{HTML}{00215E}
\colorlet{name}{black}
\colorlet{tagline}{PastelRed}
\colorlet{heading}{DarkPastelRed}
\colorlet{headingrule}{GoldenEarth}
\colorlet{subheading}{PastelRed}
\colorlet{accent}{PastelRed}
\colorlet{emphasis}{SlateGrey}
\colorlet{body}{LightGrey}

% Change some fonts, if necessary
\renewcommand{\namefont}{\Huge\rmfamily\bfseries}
\renewcommand{\personalinfofont}{\footnotesize}
\renewcommand{\cvsectionfont}{\LARGE\rmfamily\bfseries}
\renewcommand{\cvsubsectionfont}{\large\bfseries}


% Change the bullets for itemize and rating marker
% for \cvskill if you want to
\renewcommand{\cvItemMarker}{{\small\textbullet}}
\renewcommand{\cvRatingMarker}{\faCircle}
% ...and the markers for the date/location for \cvevent
% \renewcommand{\cvDateMarker}{\faCalendar*[regular]}
% \renewcommand{\cvLocationMarker}{\faMapMarker*}


% If your CV/résumé is in a language other than English,
% then you probably want to change these so that when you
% copy-paste from the PDF or run pdftotext, the location
% and date marker icons for \cvevent will paste as correct
% translations. For example Spanish:
% \renewcommand{\locationname}{Ubicación}
% \renewcommand{\datename}{Fecha}


%% Use (and optionally edit if necessary) this .tex if you
%% want to use an author-year reference style like APA(6)
%% for your publication list
% % When using APA6 if you need more author names to be listed
% because you're e.g. the 12th author, add apamaxprtauth=12
\usepackage[backend=biber,style=apa6,sorting=ydnt]{biblatex}
\defbibheading{pubtype}{\cvsubsection{#1}}
\renewcommand{\bibsetup}{\vspace*{-\baselineskip}}
\AtEveryBibitem{%
  \makebox[\bibhang][l]{\itemmarker}%
  \iffieldundef{doi}{}{\clearfield{url}}%
}
\setlength{\bibitemsep}{0.25\baselineskip}
\setlength{\bibhang}{1.25em}


%% Use (and optionally edit if necessary) this .tex if you
%% want an originally numerical reference style like IEEE
%% for your publication list
\usepackage[backend=biber,style=ieee,sorting=ydnt,defernumbers=true]{biblatex}
%% For removing numbering entirely when using a numeric style
\setlength{\bibhang}{1.25em}
\DeclareFieldFormat{labelnumberwidth}{\makebox[\bibhang][l]{\itemmarker}}
\setlength{\biblabelsep}{0pt}
\defbibheading{pubtype}{\cvsubsection{#1}}
\renewcommand{\bibsetup}{\vspace*{-\baselineskip}}
\AtEveryBibitem{%
  \iffieldundef{doi}{}{\clearfield{url}}%
}


% %% sample.bib contains your publications
% \addbibresource{sample.bib}

\begin{document}
\name{Amaury Bilocq}
\tagline{Aerospace Engineer}
%% You can add multiple photos on the left or right
%\photoR{5cm}{photo.png}

\photoL{4.5cm}{photo.png}

\personalinfo{%
  % Not all of these are required!
  \email{amaury.bilocq@gmail.com}
  %\phone{000-00-0000}
  %\mailaddress{Allée de la découverte, 9 4000 Liège, Belgium}
  \location{Liège, Belgium}\\
  %\twitter{@twitterhandle}
  \linkedin{amaury-bilocq-a42a97146/}\\
  \homepage{orbi.uliege.be/profile?uid=p216754}
  %\orcid{0000-0000-0000-0000}
  %% You can add your own arbitrary detail with
  %% \printinfo{symbol}{detail}[optional hyperlink prefix]
  % \printinfo{\faPaw}{Hey ho!}[https://example.com/]

  %% Or you can declare your own field with
  %% \NewInfoFiled{fieldname}{symbol}[optional hyperlink prefix] and use it:
  % \NewInfoField{gitlab}{\faGitlab}[https://gitlab.com/]
  % \gitlab{your_id}
  %%
  %% For services and platforms like Mastodon where there isn't a
  %% straightforward relation between the user ID/nickname and the hyperlink,
  %% you can use \printinfo directly e.g.
  % \printinfo{\faMastodon}{@username@instace}[https://instance.url/@username]
  %% But if you absolutely want to create new dedicated info fields for
  %% such platforms, then use \NewInfoField* with a star:
  % \NewInfoField*{mastodon}{\faMastodon}
  %% then you can use \mastodon, with TWO arguments where the 2nd argument is
  %% the full hyperlink.
  % \mastodon{@username@instance}{https://instance.url/@username}
}

\makecvheader
%% Depending on your tastes, you may want to make fonts of itemize environments slightly smaller
% \AtBeginEnvironment{itemize}{\small}

%% Set the left/right column width ratio to 6:4.
\columnratio{0.6}

\begin{paracol}{2}
\cvsection{Education}

\cvevent{Ph.D.\ in Aerospace Engineering}{University of Liège}{October 2020 -- Ongoing}{Liège, Belgium}

\divider

\cvevent{M.Sc.\ in Aerospace Engineering}{University of Liège}{2020} {Liège, Belgium}

\divider

\cvevent{M.Sc.\ in Industrial Engineering (Automation)}{Henallux-Pierrard}{2018}{Virton, Belgium}

\switchcolumn
\cvsection{About me}

\begin{quote}
% Aerospace engineer with expertise in compressible aerodynamics, turbulence modeling, numerical methods and software development. Experienced in both low- and high-fidelity CFD, turbomachinery, and multidisciplinary collaboration. Passionate about automation, coding, and solving complex problem.
Aerospace engineer with expertise in simulation, software development, and fluid dynamics. Skilled in analytical thinking and understanding complex physical systems to solve challenging engineering problems. Experienced in coding and managing technical projects in collaborative environments. Passionate about aerospace, automation, and building reliable tools.
\end{quote}

\end{paracol}

\cvsection{Experience}

\cvevent{PhD candidate}{University of Liège}{2020 -- Ongoing}{Liège, Belgium}
\textcolor{myRed}{Numerical research on high-speed turbulent flows and shock capturing methods for aerospace applications.}
\begin{itemize}[leftmargin=1.5em]
  \item Developed from scratch a massively parallel high-order discontinuous Galerkin solver (C++/Python).
  \item Investigated shock-capturing strategies for accurately resolving compressible turbulence in high-speed flows.
  \item Implemented advanced co-processing tools for statistical analysis of large simulation data.
  \item Contributed to DevOps workflows with GitLab CI/CD and Docker for code testing and deployment.
  \item Supervised master's students on research projects related to numerical methods and solver development.
\end{itemize}

\divider

\cvevent{Teaching Assistant}{University of Liège}{2021 -- 2023}{Liège, Belgium}
\textcolor{myRed}{Courses: Computational Fluid Dynamics and Flow in Turbomachines}
\begin{itemize}[leftmargin=1.5em]
  \item \href{https://www.programmes.uliege.be/cocoon/20242025/cours/AERO0030-1.html}{Computational Fluid Dynamics:} Prepared, supervised, and corrected exams; developed and delivered practical sessions on numerical methods and turbulence modeling.
  \item \href{https://www.programmes.uliege.be/cocoon/20242025/cours/MECA0032-1.html}{Flow in Turbomachines:} Provided simulation data for student projects on 3D rotor/stator blade flows in both design and off-design conditions. Conducted ParaView tutorials for post-processing and flow visualization.
\end{itemize}

\divider

\cvevent{Internship, Aircraft Design}{University of Liège}{February-August 2020}{Liège, Belgium}
\textcolor{myRed}{Development of a low-fidelity model for preliminary aircraft design}
\begin{itemize}[leftmargin=1.5em]
    \item Integrated a viscous–inviscid interaction model into an existing full potential solver.
    \item Improved aerodynamic performance predictions compared to base model.
\end{itemize}

\divider

\cvevent{Internship, Satellite Avionics Production}{LuxSpace}{February -- June 2018}{Betzdorf, Luxembourg}
\textcolor{myRed}{Design of a production cell for an integrated avionics unit}
\begin{itemize}[leftmargin=1.5em]
  \item Designed a microsatellite avionics production cell using Lean 3P methodology.
  \item Integrated Industry 4.0 principles into early-stage design workflows.
\end{itemize}

\divider

\cvevent{Internship, Mechanical Design}{Jindal Film}{September-November 2015}{Virton, Belgium}
\textcolor{myRed}{Professional immersion in mechanical design and manufacturing processes}
\begin{itemize}[leftmargin=1.5em]
  \item Translated 2D mechanical drawings of a cutting blade station for plastic film production into functional 3D CAD assemblies using Inventor (AutoCAD).
  \item Collaborated with the machine shop to produce and install the station.
\end{itemize}

%% Yeah I didn't spend too much time making all the
%% spacing consistent... sorry. Use \smallskip, \medskip,
%% \bigskip, \vspace etc to make adjustments.
\medskip

% Start a 2-column paracol. Both the left and right columns will automatically
% break across pages if things get too long.

% \cvsection{Projects}

% \cvevent{Project 1}{Funding agency/institution}{}{}
% \begin{itemize}
% \item Details
% \end{itemize}

% \divider

% \cvevent{Project 2}{Funding agency/institution}{Project duration}{}
% A short abstract would also work.

% \medskip

% \cvsection{A Day of My Life}

% % Adapted from @Jake's answer from http://tex.stackexchange.com/a/82729/226
% % \wheelchart{outer radius}{inner radius}{
% % comma-separated list of value/text width/color/detail}
% \wheelchart{1.5cm}{0.5cm}{%
%   6/8em/accent!30/{Sleep,\\beautiful sleep},
%   3/8em/accent!40/Hopeful novelist by night,
%   8/8em/accent!60/Daytime job,
%   2/10em/accent/Sports and relaxation,
%   5/6em/accent!20/Spending time with family
% }

% use ONLY \newpage if you want to force a page break for
% ONLY the current column
% \newpage

% \cvsection{Publications}

% %% Specify your last name(s) and first name(s) as given in the .bib to automatically bold your own name in the publications list.
% %% One caveat: You need to write \bibnamedelima where there's a space in your name for this to work properly; or write \bibnamedelimi if you use initials in the .bib
% %% You can specify multiple names, especially if you have changed your name or if you need to highlight multiple authors.
% \mynames{Lim/Lian\bibnamedelima Tze,
%   Wong/Lian\bibnamedelima Tze,
%   Lim/Tracy,
%   Lim/L.\bibnamedelimi T.}
% %% MAKE SURE THERE IS NO SPACE AFTER THE FINAL NAME IN YOUR \mynames LIST

% \nocite{*}

% \printbibliography[heading=pubtype,title={\printinfo{\faBook}{Books}},type=book]

% \divider

% \printbibliography[heading=pubtype,title={\printinfo{\faFile*[regular]}{Journal Articles}},type=article]

% \divider

% \printbibliography[heading=pubtype,title={\printinfo{\faUsers}{Conference Proceedings}},type=inproceedings]

% \divider

% \printbibliography[heading=pubtype,title={\printinfo{\faFile*[regular]}{Others}},type=manual]

%% Switch to the right column. This will now automatically move to the second
%% page if the content is too long.
\begin{paracol}{2}

\cvsection{Skills}

\cvtag{\faCode~C/C++}
\cvtag{\faCode~Python}
\cvtag{\faFileCode~Matlab/Simulink}

\divider\smallskip

\cvtag{\LaTeX~latex}
\cvtag{\githubsymbol~git}
\cvtag{\faLinux~Linux}

\divider\smallskip

\cvtag{\faDrawPolygon~AutoCAD/Fusion 360}
\cvtag{\faDrawPolygon~Siemens NX}\\
\cvtag{\faRobot~TIA Portal}
\cvtag{\faSatellite~GMAT}

\divider\smallskip

\cvtag{\faCode~OpenFOAM}
\cvtag{\faCode~SU2}
\cvtag{\faCode~GMSH}
\cvtag{\faCode~ParaView}

\switchcolumn

\cvsection{Languages}

\cvachievement{\faLanguage}{French}{Native speaker}

\divider

\cvachievement{\faLanguage}{English}{Full professional proficiency}

\divider

\cvachievement{\faLanguage}{German}{Beginner}



% \cvsection{Referees}

% % \cvref{name}{email}{mailing address}
% \cvref{Prof.\ V.E. Terrapon}{University of Liege}

% \divider

% \cvref{Prof.\ G. Dimitriadis}{University of Liege}


\end{paracol}


\end{document}
